\documentclass[10pt,landscape]{article}
\usepackage{multicol, paralist}
\usepackage[T1]{fontenc}
\usepackage[utf8]{inputenc}
\usepackage[spanish]{babel}
\usepackage[landscape, scale=0.9]{geometry}
\usepackage{amsmath,amsthm,amsfonts,amssymb}

\pdfinfo{
  /Title (cheatsheet.pdf)
  /Creator (TeX)
  /Producer (pdfTeX 1.40)
  /Author (lecarrera)
  /Subject (CoronaTec)
  /Keywords (pdflatex, latex,pdftex,tex)}

% Turn off header and footer
\pagestyle{empty}

% Redefine section commands to use less space
\makeatletter
\renewcommand{\section}{\@startsection{section}{1}{0mm}%
                                {-1ex plus -.5ex minus -.2ex}%
                                {0.5ex plus .2ex}%x
                                {\normalfont\large\bfseries}}
\renewcommand{\subsection}{\@startsection{subsection}{2}{0mm}%
                                {-1explus -.5ex minus -.2ex}%
                                {0.5ex plus .2ex}%
                                {\normalfont\normalsize\bfseries}}
\renewcommand{\subsubsection}{\@startsection{subsubsection}{3}{0mm}%
                                {-1ex plus -.5ex minus -.2ex}%
                                {1ex plus .2ex}%
                                {\normalfont\small\bfseries}}
\makeatother

% Don't print section numbers
\setcounter{secnumdepth}{0}

\setlength{\parindent}{0pt}
\setlength{\parskip}{0pt plus 0.5ex}

%My Environments
\newtheorem{ejem}[section]{Ejemplo}
% -----------------------------------------------------------------------

\begin{document}
\raggedright
\footnotesize
\begin{multicols}{3}


% multicol parameters
% These lengths are set only within the two main columns
%\setlength{\columnseprule}{0.25pt}
\setlength{\premulticols}{1pt}
\setlength{\postmulticols}{1pt}
\setlength{\multicolsep}{1pt}
\setlength{\columnsep}{2pt}

\begin{center}
     \Large{Funciones de CoronaTec} \\
\end{center}

\section{Preguntas}

\subsection{Encabezado}
\begin{verbatim}
<tipo = encabezado[, cifras = _int_]>

<variables> ?
_v1[, vn]*_ = _expresion_ +

<pregunta>
_texto de la pregunta_ @exp
\end{verbatim}

\subsection{Selección única}
  Tambi\'en acepta \verb|cifras = _int_|. \vspace*{-1.5ex}
\begin{verbatim}
<tipo = seleccion unica[, opcion = (i[ & i]* | todos)]>

<variables> ?
_v1[, vn]*_ = _expresion_ +

<pregunta>
_texto de la pregunta_ @exp

<item> +
_texto de respuesta o distractor_ @exp
\end{verbatim}

\subsection{Respuesta corta}
\begin{verbatim}
<tipo = respuesta corta[, cifras = _int_]>

<variables> ?
_v1[, vn]*_ = _expresion_ +

<pregunta>
_texto de la pregunta_ @exp

<item[, error = _error_][, factor = _entre_0_y_1_]> +
_respuesta_
\end{verbatim}

  \section{@-expresi\'on}

\begin{asparaitem}
  \item \verb|@<_expr_>|
  \item \verb|@{_expr_}|
  \item \verb|@(_expr_)|
  \item \verb|@[_expr_]|
  \item Una \verb|@| seguida de cualquier símbolo, que es el mismo que se utiliza para cerrar, por ejemplo podría ser \verb/@|_expr_|/, siempre y cuando el s\'imbolo no se encuentre en la expresi\'on. 
\end{asparaitem}

\subsection{Operadores}
\begin{asparaitem}
    \item \verb|+ - * / // % **| \\
  Suma, resta, multiplicación, división, división entera, residuo y potencia.
\end{asparaitem}

\subsection{Expresiones lambda}
\verb|f = lambda x,y,z,... : f(x,y,z,..)|

\subsection{Funciones aleatorias}
Solamente se utilizan a la hora de definir variables.

\begin{asparaitem}
  \item \verb|randrange(start = 0, stop)|
  \item \verb|randrange(start, stop, step = 1)| \\
\quad Entero \verb|n = start + k * step|, \verb|k| entero no negativo tal que \verb|start <= n < stop|.
  \smallskip

\item \verb|randint(a,b)| \\
  \quad Entero \verb|n| en el conjunto $\{a, a+1, \dots, b\}$.
  \smallskip

\item   \verb|choice(seq)| \\
\quad Un elemento del iterable no vacío \verb|seq|.
  \smallskip

\item   \verb|sample(seq, k)| \\
\quad Muestra no ordenada de tamaño \verb|k| de un iterable. Si \verb|k| es el mismo tamaño del iterable, devuelve una permutación del mismo.
  \smallskip

\item   \verb|random()| \\
\quad Flotante con distribución uniforme en $[0, 1)$.
  \smallskip

\item   \verb|uniform(a, b)| \\
  \quad Flotante con distribución uniforme en \verb|[a,b]|.
  \smallskip

\item   \verb|gauss(mu, sigma)| \\
\quad Un elemento \verb|x| que sigue una distribución gaussiana con media \verb|mu| y desviación estándar \verb|sigma|.
\end{asparaitem}

\subsection{Sucesiones}
Toda tupla, lista o conjunto, es 0-indexado.
\begin{asparaitem}
  \item \verb|range(<stop>)|
  \item \verb|range(<start>, <stop>[, <step>])| \\
\quad El inicio predeterminado es 0. El valor es siempre menor que \verb|stop|.

\item Lista al concatanear dos o m\'as sucesiones: \\
\quad \verb|ls = [*range(...), *range(...)]|.

  \item Listas por comprensi\'on:\\
    \quad \verb|ls = [i**2 for i in range(1, 11) if i % 5]| \\
    \qquad R/ \verb|[1, 4, 9, 16, 36, 49, 64, 81]|
\end{asparaitem}

\section{Funciones de Python}

\begin{asparaitem}
\item \verb|len(xs)|, \verb|sorted(xs)| \\
\quad Longitud de \verb|xs|; devuelve \verb|xs| ordenada.

\item \verb|round(x[, ndigits=None])|  \\
\quad  Redondea un número, y se puede especificar el número de dígitos a utilizar. Ninguno es el valor predeterminado.

    \item \verb|sum(ls)| \\
\quad Suma de los elementos de \verb|ls|.
\end{asparaitem}

\subsection{Texto}

\begin{asparaenum}
  \item Si \verb|num| es un entero, se puede imprimir el entero con signo como: \verb|'%+d' % num|
  \item Si \verb|x| es un número con decimales:
    \begin{itemize}
      \item \verb|'%f' % x| imprime el número con 6 decimales de manera predeterminada.
      \item \verb|'%.nf' % x|, donde \verb|n| es un entero positivo, imprime el número con \verb|n| decimales.
      \item \verb|'%e' % x| imprime el número en notación científica con 6 decimales de forma predeterminada.
      \item \verb|'%.ne' % x|, donde \verb|n| es un entero positivo, imprime el número en notación científica con \verb|n| decimales.
    \end{itemize}
  \item Si se necesita concatenar varios elementos: \verb|'(%d, %d)' % (m, n)| representa un par ordenado de enteros, o por ejemplo: \verb|'Utilice el método %s para %s.' % (var1, var2)| donde \verb|var1| y \verb|var2| son variables de texto.
\end{asparaenum}

\subsection{Conjuntos}

\begin{asparaitem}
\item Se construyen utilizando valores entre llaves, o con el comando \verb|set| sobre una lista.
\begin{verbatim}
s1 = set(sample(range(1, 16), 5))
s2 = {3, 5, 7, 11}
\end{verbatim}

\item \verb|ss.difference(s1, s2, ...)|; \verb|ss.intersection(s1, s2, ...)|; \verb|ss.union(s1, s2, ...)| \\
 \quad Se resta a \verb|ss| cada uno de los conjuntos $s_i$ y se devuelve el resultado; devuelve la intersección de los conjuntos \verb|ss|, \verb|s1|, \dots; y lo mismo para la unión.

\item  \verb|ss.symmetric_difference(s1)| \\
 \quad Diferencia simétrica de \verb|ss| con \verb|s1|.

\item  \verb|ss.isdisjoint(s1)|; \verb|ss.issubset(s1)| \\
  \quad \verb|True| si \verb|ss| y \verb|s1| son disjuntos o si \verb|ss| es subconjunto de \verb|s1|.
\end{asparaitem}

\subsection{Biblioteca \texttt{math}}

\begin{asparaitem}
    \item \verb|math.degrees(x)|; \verb|math.radians(x)| \\
      \quad Conversión de radianes a grados; y de grados a radianes.

\item \verb|math.acos(x)|;
      \verb|math.acosh(x)|;
      \verb|math.asin(x)|;
      \verb|math.asinh(x)|;
      \verb|math.atan(x)|;
      \verb|math.atan2(y, x)|;
      \verb|math.atanh(x)|;
      \verb|math.cos(x)|;
      \verb|math.cosh(x)|;
      \verb|math.sin(x)|;
      \verb|math.sinh(x)|;
      \verb|math.tan(x)|;
      \verb|math.tanh(x)| \\
\quad Funciones trigonométricas y sus inversas (en radianes).

\item \verb|math.isqrt(x)|; \verb|math.sqrt(x)|; \verb|math.exp(x)|; \verb|math.log(x)|; \verb|math.log10(x)| \\
\quad Parte entera de la raíz de \verb|x|; funciones en punto flotante para la raíz, exponencial, logaritmo natural y logaritmo en base 10.

    \item \verb|math.erf(x)|; \verb|math.erfc(x)|; \verb|math.gamma(x)| \\
\quad Funciones de error y función gamma.

  \item \verb|math.gcd(a,b)| \\
 \quad Máximo común divisor de $a$ y $b$.

    \item \verb|math.fmod(x, y)|; \verb|math.modf(x) -> (float, float)| \\
\quad Función residuo para variables de tipo flotante; parte decimal y parte entera de $x$.

    \item \verb|math.ceil(x)|;
      \verb|math.floor(x)|;
      \verb|math.trunc(x)| \\
\quad Funciones de manejo de decimales.

    \item \verb|math.factorial(a)|; \verb|math.comb(n, r)|; \verb|math.perm(n, r)| \\
\quad Factorial, combinaciones y permutaciones.% Para la respuesta, los estudiantes pueden utilizar respectivamente \verb|factorial(n), C(n, r) y P(n, r)|. Se aconseja separar con un espacio los argumentos, para poder diferenciar de aquellas respuestas en que se use coma decimal.

    \item \verb|math.hypot(x, y)|; \verb|math.dist(xs, ys)| \\
\quad $\sqrt{x^2+y^2}$ y distancia entre los vectores $n$-dimensionales  \verb|xs| y \verb|ys|

    \item \verb|math.prod(ls)| \\
\quad Producto de los elementos de \verb|ls|.
\end{asparaitem}


\newpage
\section{Funciones CoronaTec}
Funciones creadas en el proyecto. Se pueden ir agregando seg\'un se requiera.

\subsection{Biblioteca \texttt{mate}}
\begin{asparaitem}
  \item \verb|mate.factores(a)| \\
\quad Factorización de $a$. Por ejemplo \verb|mate.factores(1000)| devuelve la lista \verb|[(2, 3), (5, 3)]|, que representa a $2^3\cdot 5^3$.

\item \verb|mate.descomponer(f)| \\
  \quad Dado un n\'umero flotante $f$, lo descompone en su mantisa y su exponente. Devuelve un par ordenado, donde el primer valor es un flotante en el intervalo $[1, 10)$, y el segundo es un entero.

\item \verb|descUnid(f)| \\
  \quad Dado un n\'umero flotante $f$, lo descompone en un valor y unidades. Devuelve un par ordenado, donde el primer valor es un flotante en el intervalo $[1, 1000)$, y el segundo es una cadena de texto a utilizar.
\end{asparaitem}

\subsection{Biblioteca \texttt{txt}}
Para imprimir en la pregunta, o en las opciones de una pregunta de selecci\'on \'unica. Texto en \LaTeX. Se asume que se est\'a en modo matem\'atico.
\begin{asparaitem}
\item \verb|txt.coef(a: int, conSigno: bool = False)| \\
\quad Coeficiente que precede a una variable entera. Si $a=1$ entonces no se escribe (o se escribe solo un signo + si \verb|conSigno|). De manera similar ocurre si $a=-1$. En caso contrario imprime el valor.

\item \verb|txt.conSigno(n: int)| \\
\quad Imprime el entero con signo.

\item \verb|txt.decimal(x: float, cifras: int, conSigno = False)| \\
\quad Imprime un flotante según el n\'umero de cifras significativas indicado.

\item \verb|txt.expo(n: int)| \\
\quad Si \verb|n == 1| no imprime nada. Si no imprime \verb|^{n}|.

\item \verb|txt.fraccion(num: int, den: int, conSigno=False,| \\
      \verb|             signoNum=False, dfrac=True, coef=False)| \\
\quad Simplifica e imprime utilizando \verb|dfrac|. De manera opcional se especifica si se obliga el signo \verb|+|; si el signo se imprime en el numerador o se imprime afuera; si se utiliza \verb|frac| en lugar de \verb|dfrac|, y si es un coeficiente, de manera que si da 1 o $-1$ no se imprima el valor.

\item  \verb|txt.raiz(a: int, n: int = 2, conSigno = False)| \\
\quad Simplifica y escribe la raíz respectiva. De manera opcional se puede especificar el \'indice \verb|n|, y si se imprime un \verb|+| al inicio.

\item \verb|txt.texto(n: int)| \\
\quad Para $n > 0$, devuelve el texto que representa al valor de $n$.
\end{asparaitem}

\subsection{Biblioteca \texttt{vector}}
\begin{asparaitem}
\item \verb|vector.ceros(n: int)| \\
\quad Vector de ceros de tamaño \verb|n|. 

\item \verb|vector.aleatorio(n: int, vmin: int, vmax: int,| \\
      \verb|                   factor: float = 1)| \\
  \quad Vector de números aleatorios. Genera inicialmente enteros en \verb|vmin..vmax|, y luego los multiplica por \verb|factor|.

\item \verb|vector.latex(v: Vector, txtSep: str,| \\
      \verb|               ndigits: int = -1, ceros: int = 3)| \\
  \quad Imprime un vector. El encabezado en \LaTeX\ lo especifica el usuario. El texto separador da el formato entre elementos: por ejemplo \verb|', '| o \verb|' & '| (para vectores fila), o \verb|' \\\\ '| (para vector columna). De manera opcional se puede especificar el n\'umero de d\'igitos a imprimir. Si es $-1$, intenta imprimir la menor cantidad \verb|n|, tal que \verb|round(10**n * v[i], ceros)| sea un entero.
\end{asparaitem}

\subsection{Biblioteca \texttt{matriz}}

\begin{asparaitem}
\item \verb|matriz.aleatorio(nfilas: int, ncols: int, vmin: int,| \\
      \verb|                 vmax: int, factor: float = 1)| \\
  \quad Matriz de valores aleatorios. Ver \verb|vector.aleatorio|

\item \verb|matriz.latex(mat: Matriz, cifras = -1, ceros = 3)| \\
  \quad Imprime una matriz. El encabezado \LaTeX\  lo especifica el usuario.

\item \verb|matriz.intercambiar(mat, fila1: int, fila2: int)| \\
  \quad Matriz resultante al intercambiar dos filas de una matriz.

\item \verb|matriz.permutar(A: Matriz, perm: List[int])| \\
\quad Coloca las filas de A, según la permutación dada.

\item \verb|matriz.dominante(n, vmin, vmax, factor = 1)| \\
    \quad Matriz cuadrada diagonalmente dominante. El rango de valores es solamente para los elementos fuera de la diagonal.

\item \verb|matriz.jacobi(A: Matriz, bb: v.Vector, x0: v.Vector,|\\
      \verb|                npasos: int)| \\
      \quad Vector que se obtiene al aplicar \verb|npasos| de Jacobi.

\item \verb|matriz.gaussSeidel(A, bb, x0, npasos)|\\
  \quad Vector que se obtiene al aplicar \verb|npasos| de Gauss-Seidel.

\item \verb|matriz.sistema(A, bb)|\\
  \quad Resuelve el sistema. Devuelve una lista de valores.
\end{asparaitem}


\subsection{Biblioteca \texttt{metodos}}


\begin{asparaitem}
\item \verb|cero(f, a, b)| \\
  Determina un cero de la funci\'on $f$ en el intervalo $(a, b)$. Asume que $f(a)\cdot f(b) < 0$.

\item \verb|cuadratica(a, b, c)|\\
  \quad Determina, para un discriminante no-negativo, las soluciones reales.

\item \verb|integral(f, a, b, eps = 1e-12, prof = math.inf)| \\
  \quad M\'etodo adaptativo y recursivo que aproxima la integral de $f$ entre $a$ y $b$. \verb|eps| es el m\'aximo error relativo aceptado y \verb|prof| la m\'axima profundidad en la recursividad.

\item \verb|newton(f, fp, x0, nmax = math.inf, eps = 1e-16)|\\
  \quad M\'etodo de Newton, donde se da la funci\'on, la derivada, el valor inicial, y opcionalmente el m\'aximo n\'umero de iteraciones y tal que $|f(x)|<\varepsilon$, donde $x$ es la respuesta.

\item \verb|fmin(f, a, b, tries = 3, eps = 1e-6, delta = 0.5)| \\
  \quad M\'etodo que intenta encontrar el m\'inimo de f en el intervalo $[a, b]$. \verb|tries| se refiere al n\'umero de intentos \emph{sin} que haya mejorado la soluci\'on. En cada caso eval\'ua el doble de puntos que en el caso anterior. Utiliza el valor de \verb|delta| para construir una cuadr\'atica y determinar el v\'ertice: $\delta = \min(\delta, (x_{i+1}-x_i)/2)$. \verb|eps| se refiere a la diferencia en t\'erminos de \verb|delta|, tal que un nuevo v\'ertice se considere lo suficientemente cercano para detenerse. Devuelve $(x^*, f(x^*))$.

\end{asparaitem}



% You can even have references
\rule{0.3\linewidth}{0.25pt}
\scriptsize \\
Luis Ernesto Carrera Retana \\
\today
\end{multicols}
\end{document}

