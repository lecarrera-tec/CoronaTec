\documentclass[10pt,landscape]{article}
\usepackage{multicol, paralist}
\usepackage[T1]{fontenc}
\usepackage[utf8]{inputenc}
\usepackage[spanish]{babel}
\usepackage[landscape, scale=0.9]{geometry}
\usepackage{amsmath,amsthm,amsfonts,amssymb}

\pdfinfo{
  /Title (cheatsheet.pdf)
  /Creator (TeX)
  /Producer (pdfTeX 1.40)
  /Author (lecarrera)
  /Subject (CoronaTec)
  /Keywords (pdflatex, latex,pdftex,tex)}

% Turn off header and footer
\pagestyle{empty}

% Redefine section commands to use less space
\makeatletter
\renewcommand{\section}{\@startsection{section}{1}{0mm}%
                                {-1ex plus -.5ex minus -.2ex}%
                                {0.5ex plus .2ex}%x
                                {\normalfont\large\bfseries}}
\renewcommand{\subsection}{\@startsection{subsection}{2}{0mm}%
                                {-1explus -.5ex minus -.2ex}%
                                {0.5ex plus .2ex}%
                                {\normalfont\normalsize\bfseries}}
\renewcommand{\subsubsection}{\@startsection{subsubsection}{3}{0mm}%
                                {-1ex plus -.5ex minus -.2ex}%
                                {1ex plus .2ex}%
                                {\normalfont\small\bfseries}}
\makeatother

% Don't print section numbers
\setcounter{secnumdepth}{0}

\setlength{\parindent}{0pt}
\setlength{\parskip}{0pt plus 0.5ex}

%My Environments
\newtheorem{ejem}[section]{Ejemplo}
% -----------------------------------------------------------------------

\begin{document}
\raggedright
\footnotesize
\begin{multicols}{3}


% multicol parameters
% These lengths are set only within the two main columns
%\setlength{\columnseprule}{0.25pt}
\setlength{\premulticols}{1pt}
\setlength{\postmulticols}{1pt}
\setlength{\multicolsep}{1pt}
\setlength{\columnsep}{2pt}

\begin{center}
     \Huge{CoronaTec} \\
\end{center}

\section{Preguntas}

\subsection{Encabezado}
\begin{verbatim}
<tipo = encabezado[, cifras = _int_]>

<variables> ?
_v1[, vn]*_ = _expresion_ +

<pregunta>
_texto de la pregunta_ @exp
\end{verbatim}

\subsection{Selección única}
  Tambi\'en acepta \verb|cifras = _int_|. Los \'indices para \verb|opcion| son 0-indexados.\vspace*{-1.5ex}
\begin{verbatim}
<tipo = seleccion unica[, opcion = (i[ & i]* | todos)]>

<variables> ?
_v1[, vn]*_ = _expresion_ +

<pregunta>
_texto de la pregunta_ @exp

<item> +
_texto de respuesta o distractor_ @exp
\end{verbatim}

Es posible utilizar un comod\'in. Lo que hace es definir la opci\'on o las opciones seg\'un un string en las variables.
\verb|<tipo = seleccion unica, comodin=var>|
Dicho comod\'in deber\'ia ser 0 usualmente, pero por alg\'un error, podr\'ia tomar otro valor o valores igual que en el caso de \verb|opcion|.

\subsection{Respuesta corta}
\begin{verbatim}
<tipo = respuesta corta[, cifras = _int_]>
<tipo = respuesta corta[, funcion=_txtfun_]>

<variables> ?
_v1[, vn]*_ = _expresion_ +

<pregunta>
_texto de la pregunta_ @exp

<item[, error = _error_][, factor = _entre_0_y_1_]> +
_respuesta_
\end{verbatim}

Puede utilizar \verb|error = inf| en caso de que deba aceptar todas las respuestas como correctas. Lo \'unico es que el sistema se la pone buena \'unicamente en los casos en que respondieron la pregunta. Si no respondieron, el sistema le pone un 0.

Digamos que la respuesta es \verb|alfa|, pero en algunos casos, por un error de la pregunta, no es posible determinar el valor. As\'i, puede usar:
\begin{verbatim}
<variables>
...
alfa = math.nan if <sin_solucion> else <casos_buenos>
...

<item>
alfa
\end{verbatim}

Cuando se est\'a calificando la prueba, y el sistema encuentre un valor de \verb|math.nan| en la respuesta dada por el sistema, entonces el sistema pone buena la pregunta.

\noindent\verb|<tipo = respuesta corta[, funcion=_txtfun_]>|
Es posible pasar una funci\'on que pre-procese las respuestas. Lo \'unico es que hay que pasarla como un el texto de una funci\'on. Digamos que queremos calificar una pregunta cuya respuesta es una ecuaci\'on lineal $mx+b$.
\begin{verbatim}
<variables>
m, b = randint(1, 9)
ftxt = r"lambda ss: set(ss.replace('+', ' ').split())"
resp = {txt.coef(m) + 'x', str(b)}
...
<item, f=ftxt>
resp
\end{verbatim}
Este proceso califica buenas preguntas que se hayan escrito de la forma $mx+b$ o $b+mx$, al hacer la comparaci\'on utilizando conjuntos.

  \section{@-expresi\'on}

\begin{asparaitem}
  \item \verb|@<_expr_>|
  \item \verb|@{_expr_}|
  \item \verb|@(_expr_)|
  \item \verb|@[_expr_]|
  \item Una \verb|@| seguida de cualquier símbolo, que es el mismo que se utiliza para cerrar, por ejemplo podría ser \verb/@|_expr_|/, siempre y cuando el s\'imbolo no se encuentre en la expresi\'on. 
\end{asparaitem}

\subsection{Operadores}
\begin{asparaitem}
    \item \verb|+ - * / // % **| \\
  Suma, resta, multiplicación, división, división entera, residuo y potencia.
\end{asparaitem}

\subsection{Expresiones lambda}
\begin{verbatim}
f = lambda x,y,z : <expresion que depende de x,y,z>
r = f(x,y,z)
\end{verbatim}
Una forma de agregar funciones espec\'ificas. El problema es que debido a que Python utiliza evaluaci\'on perezosa (\emph{lazy evaluation}), no se pueden utilizar variables definidas previamente en la expresi\'on derecha de la definici\'on de $f$. En caso de que se deban utilizar variables \verb|m, n| definidas previamente, lo que se sugiere es que se agreguen como variables en la definici\'on de la funci\'on:
\begin{verbatim}
m = ...
n = ...
f = lambda u,v,x,y,z : <expresion de u,v,x,y,z>
r = f(m,n,x,y,z)
\end{verbatim}

\subsection{Funciones aleatorias}
Solamente se utilizan a la hora de definir variables.

\begin{asparaitem}
  \item \verb|randrange(start = 0, stop)|
  \item \verb|randrange(start, stop, step = 1)| \\
\quad Entero \verb|n = start + k * step|, \verb|k| entero no negativo tal que \verb|start <= n < stop|.
  \smallskip

\item \verb|randint(a,b)| \\
  \quad Entero \verb|n| en el conjunto $\{a, a+1, \dots, b\}$.
  \smallskip

\item   \verb|choice(seq)| \\
\quad Un elemento del iterable no vacío \verb|seq|.
  \smallskip

\item   \verb|choices(seq, k=1)| \\
  \quad Devuelve $k$ elementos, seleccionados \textbf{con reemplazo}.
  \smallskip

\item   \verb|sample(seq, k)| \\
\quad Muestra no ordenada de tamaño \verb|k| de un iterable. Si \verb|k| es el mismo tamaño del iterable, devuelve una permutación del mismo.
  \smallskip

\item   \verb|random()| \\
\quad Flotante con distribución uniforme en $[0, 1)$.
  \smallskip

\item   \verb|uniform(a, b)| \\
  \quad Flotante con distribución uniforme en \verb|[a,b]|.
  \smallskip

\item   \verb|gauss(mu, sigma)| \\
\quad Un elemento \verb|x| que sigue una distribución gaussiana con media \verb|mu| y desviación estándar \verb|sigma|.
\end{asparaitem}

\subsection{Sucesiones}
Toda tupla o lista, es 0-indexado.
\begin{asparaitem}
  \item \verb|range(<stop>)|
  \item \verb|range(<start>, <stop>[, <step>])| \\
\quad El inicio predeterminado es 0. El valor es siempre menor que \verb|stop|.

\item Lista al concatanear dos o m\'as sucesiones: \\
\quad \verb|ls = [*range(...), *range(...)]|.

  \item Listas por comprensi\'on:\\
    \quad \verb|ls = [i**2 for i in range(1, 11) if i % 5]| \\
    \qquad R/ \verb|[1, 4, 9, 16, 36, 49, 64, 81]|
\end{asparaitem}

\section{Funciones de Python}

\begin{asparaitem}
\item \verb|len(xs)|, \verb|sorted(xs)| \\
\quad Longitud de \verb|xs|; devuelve \verb|xs| ordenada.

\item \verb|round(x[, ndigits=None])|  \\
\quad  Redondea un número, y se puede especificar el número de dígitos a utilizar. Ninguno es el valor predeterminado.

    \item \verb|sum(ls)| \\
\quad Suma de los elementos de \verb|ls|.
\end{asparaitem}

\subsection{Texto}

\begin{asparaenum}
  \item Si \verb|num| es un entero, se puede imprimir el entero con signo como: \verb|'%+d' % num|
  \item Si \verb|x| es un número con decimales:
    \begin{itemize}
      \item \verb|'%f' % x| imprime el número con 6 decimales de manera predeterminada.
      \item \verb|'%.nf' % x|, donde \verb|n| es un entero positivo, imprime el número con \verb|n| decimales.
      \item \verb|'%e' % x| imprime el número en notación científica con 6 decimales de forma predeterminada.
      \item \verb|'%.ne' % x|, donde \verb|n| es un entero positivo, imprime el número en notación científica con \verb|n| decimales.
    \end{itemize}
  \item Si se necesita concatenar varios elementos: \verb|'(%d, %d)' % (m, n)| representa un par ordenado de enteros, o por ejemplo: \verb|'Utilice el método %s para %s.' % (var1, var2)| donde \verb|var1| y \verb|var2| son variables de texto.
\end{asparaenum}

\subsection{Conjuntos}

\begin{asparaitem}
\item Se construyen utilizando valores entre llaves, o con el comando \verb|set| sobre una lista.
\begin{verbatim}
s1 = set(sample(range(1, 16), 5))
s2 = {3, 5, 7, 11}
\end{verbatim}

\item \verb|ss.difference(s1, s2, ...)|; \verb|ss.intersection(s1, s2, ...)|; \verb|ss.union(s1, s2, ...)| \\
 \quad Se resta a \verb|ss| cada uno de los conjuntos $s_i$ y se devuelve el resultado; devuelve la intersección de los conjuntos \verb|ss|, \verb|s1|, \dots; y lo mismo para la unión.

\item  \verb|ss.symmetric_difference(s1)| \\
 \quad Diferencia simétrica de \verb|ss| con \verb|s1|.

\item  \verb|ss.isdisjoint(s1)|; \verb|ss.issubset(s1)| \\
  \quad \verb|True| si \verb|ss| y \verb|s1| son disjuntos o si \verb|ss| es subconjunto de \verb|s1|.
\end{asparaitem}

\subsection{Biblioteca \texttt{math}}

\begin{asparaitem}
    \item \verb|math.degrees(x)|; \verb|math.radians(x)| \\
      \quad Conversión de radianes a grados; y de grados a radianes.

\item \verb|math.acos(x)|;
      \verb|math.acosh(x)|;
      \verb|math.asin(x)|;
      \verb|math.asinh(x)|;
      \verb|math.atan(x)|;
      \verb|math.atan2(y, x)|;
      \verb|math.atanh(x)|;
      \verb|math.cos(x)|;
      \verb|math.cosh(x)|;
      \verb|math.sin(x)|;
      \verb|math.sinh(x)|;
      \verb|math.tan(x)|;
      \verb|math.tanh(x)| \\
\quad Funciones trigonométricas y sus inversas (en radianes).

\item \verb|math.isqrt(x)|; \verb|math.sqrt(x)|; \verb|math.exp(x)|; \verb|math.log(x)|; \verb|math.log10(x)| \\
\quad Parte entera de la raíz de \verb|x|; funciones en punto flotante para la raíz, exponencial, logaritmo natural y logaritmo en base 10.

    \item \verb|math.erf(x)|; \verb|math.erfc(x)|; \verb|math.gamma(x)| \\
\quad Funciones de error y función gamma.

  \item \verb|math.gcd(a_1,a_2,...,a_n)| \\
    \quad Máximo común divisor del conjunto $\{a_1,a_2,\dots,a_n\}$.

    \item \verb|math.fmod(x, y)|; \verb|math.modf(x) -> (float, float)| \\
\quad Función residuo para variables de tipo flotante; parte decimal y parte entera de $x$.

    \item \verb|math.ceil(x)|;
      \verb|math.floor(x)|;
      \verb|math.trunc(x)| \\
\quad Funciones de manejo de decimales.

    \item \verb|math.factorial(a)|; \verb|math.comb(n, r)|; \verb|math.perm(n, r)| \\
\quad Factorial, combinaciones y permutaciones.% Para la respuesta, los estudiantes pueden utilizar respectivamente \verb|factorial(n), C(n, r) y P(n, r)|. Se aconseja separar con un espacio los argumentos, para poder diferenciar de aquellas respuestas en que se use coma decimal.

    \item \verb|math.hypot(x, y)|; \verb|math.dist(xs, ys)| \\
\quad $\sqrt{x^2+y^2}$ y distancia entre los vectores $n$-dimensionales  \verb|xs| y \verb|ys|

    \item \verb|math.prod(ls)| \\
\quad Producto de los elementos de \verb|ls|.
\end{asparaitem}

\subsection{Biblioteca \texttt{Fraction}}
Permite representar n\'umeros como fracciones:

\verb|f = Fraction(a, b)|

y realizar operaciones aritm\'eticas con ellos, las cuales se simplifican de manera autom\'atica. Se puede extraer luego su numerator y denominador mediante \verb|f.numerator| y \verb|f.denominator|.

\section{Funciones CoronaTec}
Funciones creadas en el proyecto. Se pueden ir agregando seg\'un se requiera.

\subsection{Biblioteca \texttt{mate}}
\begin{asparaitem}
\item \verb|mate.descomponer(f)| \\
  \quad Dado un n\'umero flotante $f$, lo descompone en su mantisa y su exponente. Devuelve un par ordenado, donde el primer valor es un flotante en el intervalo $[1, 10)$, y el segundo es un entero.

\item \verb|mate.descUnid(f)| \\
  \quad Dado un n\'umero flotante $f$, lo descompone en un valor y unidades. Devuelve un par ordenado, donde el primer valor es un flotante en el intervalo $[1, 1000)$, y el segundo es una cadena de texto a utilizar.

\item \verb|mate.digSignif(x, d)| \\
  \quad Redondeo de un n\'umero de punto flotante $x$, seg\'un la cantidad de d\'igitos significativos $d$.

\item \verb|mate.divisores(n)| \\
  \quad Devuelve una lista de los divisores positivos del n\'umero entero $n$.

  \item \verb|mate.factores(a)| \\
\quad Factorización de $a$. Por ejemplo \verb|mate.factores(1000)| devuelve la lista \verb|[(2, 3), (5, 3)]|, que representa a $2^3\cdot 5^3$.

  \item \verb|mate.raiz(a, indice=2)| \\
    \quad Devuelve lo que queda afuera y lo que queda adentro de la ra\'iz como una tupla. Por ejemplo, \verb|mate.raiz(8)->(2,2)|.

\end{asparaitem}

\subsection{Biblioteca \texttt{txt}}
Para imprimir en la pregunta, o en las opciones de una pregunta de selecci\'on \'unica. Texto en \LaTeX. Se asume que se est\'a en modo matem\'atico.
\begin{asparaitem}
\item \verb|txt.coef(a: int, conSigno: bool = False, arg='')| \\
\quad Coeficiente que precede a una variable entera. Si $a=1$ entonces no se escribe (o se escribe solo un signo + si \verb|conSigno|). De manera similar ocurre si $a=-1$. En caso contrario imprime el valor. Si $a=0$ no imprime nada. Si se le incluye un argumento, lo agrega al final de la expresi\'on, esto es \'util si el coeficiente puede ser 0.

\item \verb|txt.conSigno(n: int)| \\
\quad Imprime el entero con signo.

\item \verb|txt.decimal(x: float, cifras: int, conSigno = False)| \\
\quad Imprime un flotante según el n\'umero de cifras significativas indicado.

\item \verb|txt.expo(n: int, arg='', coef=False)| \\
  \quad Si \verb|n == 1| no imprime nada. Si \verb|n==0| y \verb|arg|, imprime 1 o nada (seg\'un \verb|coef|). Si no imprime \verb|arg^{n}|.

\item \verb|txt.fraccion(num, den=1, conSigno=False,|\\
      \verb|     signoNum=False, dfrac=True, arg='', coef=False)| \\
  \quad Simplifica e imprime utilizando \verb|dfrac|. De manera opcional se especifica si se obliga el signo \verb|+|; si el signo se imprime en el numerador o se imprime afuera; si se utiliza \verb|tfrac| en lugar de \verb|dfrac|, y si tiene un argumento, de manera que si da 1 o $-1$ no se imprima el valor. Se puede utilizar \verb|Fraction| en el numerador, y ya no ser\'ia necesario especificar el denominador. Si \verb|coef=True|, entonces se imprime sólo el signo si es 1 o $-1$.

Si el argumento es distinto de vac\'io, el numerador se toma como un coeficiente.

\item  \verb|txt.raiz(a: int, n: int = 2, conSigno = False)| \\
\quad Simplifica y escribe la raíz respectiva. De manera opcional se puede especificar el \'indice \verb|n|, y si se imprime un \verb|+| al inicio.

\item \verb|txt.texto(n: int)| \\
\quad Para $n > 0$, devuelve el texto que representa al valor de $n$.
\end{asparaitem}

\subsection{Biblioteca \texttt{vector}}
\begin{asparaitem}
\item \verb|vector.ceros(n: int)| \\
\quad Vector de ceros de tamaño \verb|n|. 

\item \verb|vector.aleatorio(n: int, vmin: int, vmax: int,| \\
  \verb|                   factor: Fraction = Fraction(1,1))| \\
  \quad Vector de números aleatorios. Genera inicialmente enteros en \verb|vmin..vmax|, y luego los multiplica por \verb|factor|, que puede ser un n\'umero de punto flotante, o un n\'umero como fracci\'on.

\item \verb|vector.latex(v: Vector, txtSep: str,| \\
      \verb|               cifras: int = -2, ceros: int = 3)| \\
  \quad Imprime un vector. El encabezado en \LaTeX\ lo especifica el usuario. El texto separador da el formato entre elementos: por ejemplo \verb|', '| o \verb|' & '| (para vectores fila), o \verb|' \\\\ '| (para vector columna). De manera opcional se puede especificar el n\'umero de d\'igitos a imprimir. Si es $-1$, intenta imprimir la menor cantidad \verb|n|, tal que \verb|round(10**n * v[i], ceros)| sea un entero. Observe que $-2$ es el valor predeterminado, lo que asume que lo que va a imprimir son fracciones.

\item \verb|vector.kprod(k: float, v: Vector)| \\
\quad $k\cdot v$

\item \verb|vector.pprod(u: Vector, v: Vector)| \\
\quad $u\cdot v$ Producto punto.

\item \verb|vector.suma(v1: Vector, v2: Vector)| \\
\quad $v_1 + v_2$
\end{asparaitem}

\subsection{Biblioteca \texttt{matriz}}

\begin{asparaitem}
\item \verb|matriz.aleatorio(nfilas: int, ncols: int, vmin: int,| \\
  \verb|          vmax: int, factor: Fraction = Fraction(1,1))| \\
  \quad Matriz de valores aleatorios. Ver \verb|vector.aleatorio|

\item \verb|matriz.copia(A)| \\
  \quad Una copia de la matriz $A$. No deber\'ia ser necesario utilizar esta funci\'on.

\item \verb|matriz.det(A)| \\
  \quad Determinante de la matriz $A$. Construye una matriz triangular superior utilizando pivoteo, y multiplica los valores de la diagonal para tener el valor del determinante.

\item \verb|matriz.update(A, irow, icol, valor)| \\
  \quad Cambia el valor de una entrada de la matriz.

\item \verb|matriz.dominante(n, vmin, vmax,| \\
  \verb|                             factor = Fraction(1,1))| \\
    \quad Matriz cuadrada diagonalmente dominante. El rango de valores es solamente para los elementos fuera de la diagonal.

\item \verb|matriz.gaussSeidel(A: Matriz, bb: Vector,|
  \verb|                            x0: Vector, npasos: int)|\\
  \quad Vector que se obtiene al aplicar \verb|npasos| de Gauss-Seidel.

\item \verb|matriz.intercambiar(mat, fila1: int, fila2: int)| \\
  \quad Matriz resultante al intercambiar dos filas de una matriz.

\item \verb|matriz.jacobi(A: Matriz, bb: v.Vector, |\\
      \verb|                      x0: v.Vector, npasos: int)| \\
      \quad Vector que se obtiene al aplicar \verb|npasos| de Jacobi.

\item \verb|matriz.latex(mat: Matriz, decimal = False,|
  \verb|                   dfrac = False, espacio = '[1ex]',|
  \verb|                            cifras = -20, ceros = 3)| \\
  \quad Imprime una matriz. El encabezado \LaTeX\  lo especifica el usuario. \verb|decimal| es para imprimir enteros o con punto decimal, y si no, lo imprime con \verb|tfrac| (aunque también imprime bien los enteros si no se especifica el modo decimal). \verb|dfrac| especifica si se utiliza \verb|\dfrac| en lugar de \verb|\tfrac|. \verb|espacio| es el texto que pone al final de l\'inea. Tiene como predeterminado \verb|'[1ex]'|. Puede seleccionar tambi\'en un string vac\'io: \verb|' '|. \verb|cifras| se refiere al n\'umero de cifras significativas. Si \verb|cifras==0| asume que es un entero. Solo si \verb|cifras == -1| se utiliza el menor valor posible, y solo entonces se lee la variable \verb|ceros|, que quiere decir cu\'antos ceros consecutivos se leen para descartar el resto. Es decir, con \verb|ceros=3|, el valor 2.30005 se lee como 2.3.

\item \verb|matriz.permutar(A: Matriz, perm: List[int])| \\
\quad Coloca las filas de A, según la permutación dada.

\item \verb|matriz.pivote(L: Matriz, U:Matriz, P:Matriz,|
  \verb|             ifila: int) -> (Matriz, Matriz, Matriz)| \\
  \quad Un paso de factorizaci\'on LU con pivoteo.

\item \verb|matriz.sistema(A, bb)|\\
  \quad Resuelve el sistema. Devuelve una lista de valores.

\item \verb|matriz.trans(A)|\\
  \quad Matriz transpuesta.

\item \verb|matriz.vector(A: Matriz, v: Vector)| \\
  \quad Realiza el producto $Av$.
\end{asparaitem}


\subsection{Biblioteca \texttt{metodos}}


\begin{asparaitem}
\item \verb|metodos.cero(f, a, b)| \\
  Determina un cero de la funci\'on $f$ en el intervalo $(a, b)$. Asume que $f(a)\cdot f(b) < 0$.

\item \verb|metodos.cuadratica(a, b, c)|\\
  \quad Determina, para un discriminante no-negativo, las soluciones reales.

\item \verb|metodos.regresionLineal(xs, ys, fx=None, gy=None)|
  \quad Devuelve la tupla $(m, b)$ obtenida por regresi\'on lineal de $g(y) = m\cdot f(x) + b$.

\item \verb|metodos.derivada(f, x0, n=1, delta=1e-6)| \\
  \quad Aproximaci\'on a la $n$-\'esima derivada de la funci\'on $f$ en el punto $x_0$.

\item \verb|metodos.integral(f, a, b, eps = 1e-12,|\\
  \verb|                                    prof = math.inf)| \\
  \quad M\'etodo adaptativo y recursivo que aproxima la integral de $f$ entre $a$ y $b$. \verb|eps| es el m\'aximo error relativo aceptado y \verb|prof| la m\'axima profundidad en la recursividad.

\item \verb|metodos.newton(f, fp, x0, nmax = math.inf,|\\
  \verb|                                        eps = 1e-16)|\\
  \quad M\'etodo de Newton, donde se da la funci\'on, la derivada, el valor inicial, y opcionalmente el m\'aximo n\'umero de iteraciones y tal que $|f(x)|<\varepsilon$, donde $x$ es la respuesta.

\item \verb|metodos.fmin(f, a, b, tries = 3, eps = 1e-6,|\\
  \verb|                                        delta = 0.5)| \\
  \quad M\'etodo que intenta encontrar el m\'inimo de f en el intervalo $[a, b]$. \verb|tries| se refiere al n\'umero de intentos \emph{sin} que haya mejorado la soluci\'on. En cada caso eval\'ua el doble de puntos que en el caso anterior. Utiliza el valor de \verb|delta| para construir una cuadr\'atica y determinar el v\'ertice: $\delta = \min(\delta, (x_{i+1}-x_i)/2)$. \verb|eps| se refiere a la diferencia en t\'erminos de \verb|delta|, tal que un nuevo v\'ertice se considere lo suficientemente cercano para detenerse. Devuelve $(x^*, f(x^*))$.
\end{asparaitem}


\subsection{Biblioteca \texttt{util}}

\begin{asparaitem}
\item \verb|util.lista(f, inicio, fin, args)| \\
  Debido a las caracter\'isticas de \emph{lazy evaluation}, no es posible, por ejemplo, definir variables $a,b,c,k$, y luego construir la lista:
  \verb|ls = [(a * i**2 + b * i + c)  for i in range(ini, fin)]| El problema son los valores de $a$, $b$ y $c$ que se utilizan (el valor de $k$ no lo es). As\'i que se tuvo que implementar una funci\'on que haga lo mismo:
\begin{verbatim}
f=lambda(i,args):args[0]*i**2+args[1]*i+args[2]
ls = lista(f, ini, fin, [a, b, c])
\end{verbatim}
\end{asparaitem}

\subsection{Biblioteca \texttt{conj}}
Las siguientes operaciones trabajan sobre listas.
\begin{asparaitem}
\item \verb|conj.union(A, B)| $A\cup B$
\item \verb|conj.interseccion(A, B)| $A\cap B$
\item \verb|conj.diferencia(A, B)| $A-B$
\item \verb|conj.dsimetrica(A, B)| $A \triangle B$
\item \verb|conj.potencia(A)| $P(A)=2^A$
\item \verb|conj.producto(A, B)| $A\times B$
\item \verb|conj.impComo(A)| \\
  Imprime el conjunto $A$, que es una lista.
\end{asparaitem}

\subsection{Biblioteca \texttt{relBin}}
Una matriz en relaciones binarias es una lista de listas 0-indexada, donde cada elemento de la lista principal, corresponde a la fila.

Un gr\'afico es una lista de tuplas de las parejas de elementos 0-indexados $(i,j)$ de la matriz que son 1's.

\begin{asparaitem}
\item \verb|relBin.grafico2grafico(G, Atxt, Btxt=Atxt)|
  Dado un gr\'afico $G$ y una lista de caracteres que corresponde a los elementos del conjunto de salida $A$ (y una posible lista de caracteres que corresponde a los elementos del conjunto de salida $B$ en caso de que sean distintos), construye el string que corresponde al grafo respectivo.

\item \verb|relBin.grafico2matriz(G, nfilas, ncols=nfilas)|
  Construye una matriz dado el gr\'afico.

\item \verb|relBin.matriz2grafico(M, Atxt=[], Btxt=Atxt)|
  Construye el gr\'afico, o el string del gr\'afico si dado al menos el conjunto $A$ como una lista de caracteres.

\item \verb|relBin.esMatrizReflexiva(M)|
  Determina si la matriz representa o no una relaci\'on reflexiva.

\item \verb|relBin.esMatrizSimetrica(M)|
  Determina si la matriz representa o no una relaci\'on sim\'etrica.

\item \verb|relBin.esMatrizTransitiva(M)|
  Determina si la matriz representa o no una relaci\'on transitiva.

\item \verb|relBin.esMatrizAntisimetrica(M)|
  Determina si la matriz representa o no una relaci\'on antisim\'etrica.

\item \verb|relBin.esMatrizTotal(M)|
  Determina si la matriz representa o no una relaci\'on total.

\item \verb|relBin.esMatricesMenorIgual(M1, M2)|
  Determina si $M_1 \leq M2$.

\item \verb|relBin.matrizDominio(M, Atxt)| 
  Devuelve el dominio de $M$, seg\'un los valores (como texto) de A.

\item \verb|relBin.matrizAmbito(M, Btxt)| 
  Devuelve el dominio de $M$, seg\'un los valores (como texto) de B.

\item \verb|relBin.matrizNegar(M)|
  Construye el complemento de la matriz $M$.

\item \verb|relBin.matricesAnd(M1, M2)|
  Construye la matriz $M_1 \land M_2$.

\item \verb|relBin.matricesOr(M1, M2)|
  Construye la matriz $M_1 \lor M_2$.

\item \verb|relBin.matricesComp(M1, M2)|
  Construye la composici\'on $R_2 \circ R_1$ mediante el producto $M_1 \odot M_2$.

\item \verb|relBin.matrizTranspuesta(M)|
  Construye la matriz transpuesta.
  
\item \verb|relBin.reglaGrafico(A, B, f)|
  Construye un gr\'afico a partir de una regla $f(a,b)$, que se aplica a cada elemento $(a,b)\in A\times B$.
  
\item \verb|relBin.reglaMatriz(nfilas, ncols, f)|
  Construye una matriz a partir de una regla $f(i,j)$, que se aplica a cada elemento 0-indexado de la matriz.
\end{asparaitem}

% You can even have references
\rule{0.3\linewidth}{0.25pt}
\scriptsize \\
Luis Ernesto Carrera Retana \\
\today
\end{multicols}
\end{document}

