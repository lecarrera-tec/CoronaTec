\documentclass[12pt]{article}

\usepackage[scale=0.85]{geometry}
\usepackage{amsmath,amsthm,amssymb}
\usepackage[spanish]{babel}

\usepackage{multicol}
\usepackage{paralist}

\theoremstyle{definition}
\newtheorem{funcion}{}[section]
\newtheorem{ejem}{Ejemplo}

\newcommand{\ds}{\displaystyle}

\title{Banco de Preguntas Parametrizadas}
\author{Luis Ernesto Carrera Retana}

\begin{document}

\maketitle

\section{Información general}

Este proyecto consta de los siguientes programas:
\begin{itemize}
  \item \verb|generar| Esta es la función que genera las pruebas. Requiere dos archivos como argumentos: el primero es el archivo \verb|ppp| donde se guarda la información general de la prueba, y el segundo la carpeta con las listas de los grupos de las personas estudiantes (cada uno de ellos descargado del tec digital y convertido a CSV) o un solo archivo csv. En esta misma carpeta guarda las evaluaciones. Tiene como argumento adicional un índice (predeterminado es 0), que es un factor para la semilla. Dicho índice permite generar un nuevo grupo de exámenes a los mismos estudiantes.

  \item \verb|evaluar| Esta es la función que realiza la evaluación. Requiere el archivo \verb|ppp|, la carpeta con las listas de los estudiantes, y el archivo de las respuestas descargado de Microsoft Forms, y convertido a .csv, con `;' como separador entre columnas. Tiene como argumento adicional un índice (predeterminado es 0), que es un factor para la semilla. Debe utilizar el mismo índice que utilizó para generar los exámenes.

  \item \verb|visualizar| Esta función recibe como argumento la dirección del archivo de una pregunta, el número de ejemplos por generar, para generar un pdf de la pregunta. Esto permite revisar que la pregunta esté bien definida, y ver si genera los resultados esperados.

    Si la pregunta requiere de paquetes adicionales, de manera opcional se puede pasar como argumento un \emph{encabezado} de la pregunta que debe iniciar desde \verb|\documentclass...| y llegar hasta \verb|\begin{document}|. Debe incluir un ambiente de tipo \emph{ejer}, porque cada pregunta se encierra entre un \verb|\begin{ejer}| \dots \verb|\end{ejer}|. Puede ver la variable \verb|encabezado| en el archivo \verb|visualizar.py| como referencia.

    Cuando se imprime la vista previa de preguntas de selección, el orden de los distractores se mantiene fijo según el orden original de la pregunta y se marca con `R/' la opción correcta. En el caso de preguntas de respuesta corta, se agrega la respuesta al final de la pregunta.
    \bigskip

    Sin embargo, la funci\'on de visualizar es un poco limitada. Se sugiere, mejor, utilizar el archivo \verb|.ppp| para ir viendo c\'omo va quedando la evaluaci\'on. Algunas sugerencias son, agregar en el \verb|<Encabezado>| los siguientes comandos:

    \noindent \verb|\newcommand{\esta}{\qquad $\gets$}| \\
    Utilizar, al final de la opci\'on correcta en las preguntas de selecci\'on unica, el comando \verb|\esta|.
    \medskip

    \noindent \verb|\newcommand{\resp}[1]{\hfill{\small R/#1}}| \\
    En las preguntas de selecci\'on \'unica, imprimir la respuesta correcta.
    \medskip

    \noindent \verb|\usepackage{hyperref}| \\
    \noindent \verb|\newcommand{\desc}[1]{Descripción: #1}| \\
    Imprime información pertinente. Se utiliza, por ejemplo, \verb|\desc{\url{PATH DEL ARCHIVO}}| para saber cu\'al es el archivo origen de la pregunta.
    \medskip

    Para quitar los comandos, sin cambiar los archivos \verb|.tex|, simplemente se redefinen, por ejemplo: \\
    \noindent \verb|\renewcommand{\desc}[1]{}|
\end{itemize}

\subsection{Requisitos}
\begin{itemize}
  \item python
  \item \LaTeX (miktex o texlive)
  \item perl (no se requiere para la función \verb|visualizar|)
  \item ghostscript (no se requiere para la función \verb|visualizar|)
\end{itemize}
\verb|perl| y \verb|ghostscript| son necesarios solo si se utiliza una p\'agina por pregunta, como se explicar\'a m\'as adelante.

\subsection{Comentarios}
Puede utilizar \verb|%| o \verb|#| como comentario al inicio del renglón.

\subsection{Modificando el \LaTeX}
El encabezado predeterminado se puede visualizar en el archivo \verb|latex.py|. Recuerde respaldar cualquier cambio que haga, porque cada versión nueva va a borrar los cambios que usted haya realizado.

\section{Estructura del archivo general}
\small
\begin{verbatim}
<Escuelas>
__Lista de escuelas__

<Semestre>
__Semestre y año__

<Tiempo>
__Duración de la prueba__

<Cursos>
__Nombre de los cursos__

<Titulo>
__Título de la prueba__

<Encabezado[, crop=False][, continuo=False]>
__Paquetes, comandos nuevos, etc, para el encabezado del archivo LaTeX__

<Instrucciones>
__Instrucciones del examen__

<Seccion[, orden = aleatorio]>
  <Titulo>
  __Opcional si es sólo una sección__

  <Instrucciones>
  __Opcional si es sólo una sección__

  <Preguntas>
    <bloque>
    [puntaje = __int__,] origen = __string__[, muestra = __int__]
    </bloque>
<Fin>
\end{verbatim}
\normalsize

\begin{itemize}
  \item \verb|__Texto entre guiones__| corresponde a una descripción.
  \item Todo texto entre \verb|[ ... ]|, a excepción de las listas de python, se refiere a argumentos opcionales.
  \item Tanto el nombre del curso como el título deben ser solamente \textbf{una} línea de texto debajo de la etiqueta respectiva.
  \item Las instrucciones pueden abarcar varios renglones e incluir líneas en blanco, para que \LaTeX\ separe los párrafos. 
  \item En las instrucciones \emph{no} debe aparecer el símbolo de abrir etiquetas (\verb|<|) al inicio del renglón.
  \item No debe haber espacios en blanco entre las especificaciones de las preguntas, pero sí se permiten comentarios.
  \item El \verb|origen| de la pregunta puede ser un archivo con extensión \verb|.tex| o una carpeta, con uno o más archivos con extensión \verb|.tex|.
  \item Si la dirección es una carpeta, entonces el puntaje para cualquiera de las preguntas es el mismo. Si el puntaje no aparece, entonces el predeterminado es 1~punto.
  \item La muestra se refiere al número de preguntas que se toma de la carpeta. Si no aparece, el predeterminado es 1.
  \item El bloque de \verb|<Preguntas>| debe finalizar con una línea en blanco.
  \item De manera opcional, se pueden encerrar varias preguntas entre \verb|<bloque>| y \verb|</bloque>|. El efecto es que no va a separar las preguntas con una nueva página, y va a mantener las variables entre las preguntas. Utilizar aleatorio en la sección no afecta el orden de las preguntas en el bloque.
\end{itemize}

\section{Configurando formato del documento}
En el encabezado, definir: \\
\verb|<Encabezado, crop=False>|
si lo que quiere es tener las preguntas de manera continua.
\medskip

\noindent \verb|\usepackage[fontsize=11]{fontsize}| \\
El sistema utiliza 12pt como predeterminado para el tama\~no de de letra. Este comando permite utilizar una letra m\'as peque\~na.
\medskip

\noindent \verb|\usepackage{multicol}| \\
Esto permite imprimir las opciones de las preguntas de selecci\'on \'unica en columnas. Cuando se agregue la pregunta en la secci\'on de \verb|<Preguntas>| del \verb|.ppp|, es posible especificar, en cada pregunta, las columnas deseadas: \\
\verb|puntaje = 2, columnas = 4, origen = ...|
\medskip

\noindent \verb|geometry{...}| \\
  Esto permite cambiar la geometr\'ia de la p\'agina. Ver la documentaci\'on del paquete \verb|geometry|.
\medskip

\noindent \verb|\renewcommand\thesection{\Roman{section}}| \\
  Esto permite cambiar la enumeraci\'on de las secciones, de n\'umeros a n\'umeros romanos en may\'usculas.

\noindent \verb|\renewcommand{\LatexNuevaPregunta}{...}| \\
Esta definici\'on permite cambiar el comando a utilizar para el espacio entre preguntas.
\medskip

\noindent \verb|\renewcommand{\LatexNuevaPreguntaBloque}{...}| \\
Esta definici\'on permite cambiar el comando a utilizar para el espacio entre preguntas que se encuentran en un bloque.
\medskip

\noindent \verb|\renewcommand{\FormatoNombre}[1]{...}| \\
Esta definici\'on permite cambiar el formato del nombre a imprimir en el encabezado en el examen. Por ejemplo: \\
\verb|\renewcommand{\FormatoNombre}[1]{{\Large \textbf{#1}}}|
\medskip


\section{Tipos de Preguntas}

\subsection{Encabezado}
No es una pregunta. Es más bien como un texto para un grupo de preguntas. Lo ideal es que vaya al inicio de un bloque. Se distingue en el archivo \verb|.ppp| porque tiene un puntaje de 0.

\small
\begin{verbatim}
<tipo = encabezado[, cifras = __cifras significativas__]>

<variables>
__nombre_de_variables__ = __expresion__

<pregunta>
__texto de la pregunta__
\end{verbatim}
\normalsize

Las cifras significativas se utilizan en caso de que se deba imprimir un número flotante en el enunciado, o más adelante, en los items de selección única. El valor predeterminado es 3. Ver \ref{formato} para mayor control.

\subsection{Selección única}

\small
\begin{verbatim}
<tipo = seleccion unica[, opcion = (__indice__[& __indice__]* | todos)][, cifras = ...]>
<tipo = seleccion unica, comodin = __variable__[, cifras = ...]>

<variables>
__nombre_de_variables__ = __expresion__

<pregunta>
__texto de la pregunta__

<item>
__respuesta o distractor__

\end{verbatim}
\normalsize

\begin{itemize}
  \item Es opcional especificar el índice de la respuesta (0-indexado); sino se asume que la respuesta correcta es el primero de los items.
  \item Para el caso en que haya habido un error, se puede poner \verb|opcion = todos|, es decir, que toda respuesta es correcta, o items específicos (0-indexados).
  \item En lugar de \verb|opcion|, se puede utilizar \verb|comodin|. En este caso, se define mediante una de las variables, que tiene entonces que tomar el valor de 0 hasta $n-1$, donde $n$ es el n\'umero de opciones. Se pueden usar varias opciones separadas por \verb| & |, o incluso se puede usar el valor \verb|'todos'|.
  \item \textbf{Se debe dejar una línea en blanco al final de cada item}.
  \item Las variable son opcionales.
\end{itemize}


%% \subsection{Selección única}
%% Las opciones para selección única son las siguientes:
%% \begin{itemize}
%%   \item \verb|<tipo = seleccion unica[, orden = aleatorio][, opcion = __indice__]>|
%%   \item \verb|<tipo = seleccion unica, orden = creciente[, opcion = __indice__>]|
%%   \item \verb|<tipo = seleccion unica, orden = fijo>|
%%   \item \verb|<tipo = seleccion unica, orden = indice>|
%% \end{itemize}
%% 
\subsection{Respuesta corta}

\small
\begin{verbatim}
<tipo = respuesta corta[, cifras = ...]>
<tipo = respuesta corta[, funcion=_txtfun_]>

<variables>
__nombre_de_variables__ = __expresion__

<pregunta>
__texto de la pregunta__

<item[, error = __error__][, factor = __porcentaje de la pregunta__]>
__respuesta__
\end{verbatim}
\normalsize

\begin{itemize}
  \item Para el caso en que haya habido un error, se puede poner \verb|opcion = todos|, es decir, que toda respuesta es correcta, pero no imprime la solución al ejercicio, sino que imprime un \verb|*| en el informe al estudiante. La otra opción es poner \verb|error = inf|, lo cual pone la respuesta correcta, y le pone bueno a todo estudiante que haya dado un valor como respuesta. Lo \'unico es que el sistema se la pone buena \'unicamente en los casos en que respondieron la pregunta. Si no respondieron, el sistema le pone un 0.
  \item La respuesta \textbf{no} se escribe como una @-expresión.
  \item Aunque se asume que el primer item contiene la respuesta correcta, se puede especificar más de una respuesta.  
  \item Si no se da el error, se toma que es 0. Aunque sería lo esperable en una pregunta cuya respuesta es un entero, no se recomienda en absoluto para una respuesta de tipo flotante. 
  \item Si no se da un factor, entonces se toma 1 como factor. Se utiliza para ir relajando el error, pero castigar entonces el puntaje de la pregunta. Cada radio del error debe ser mayor, y cada puntaje no debe ser mayor al anterior, porque se toma como correcto el primero que se encuentre.
  \item Digamos que la respuesta es \verb|alfa|, pero en algunos casos, por un error de la pregunta, no es posible determinar el valor. As\'i, puede usar:
\begin{verbatim}
<variables>
...
alfa = math.nan if <sin_solucion> else <casos_buenos>
...

<item>
alfa
\end{verbatim}

Cuando se est\'a calificando la prueba, y el sistema encuentre un valor de \verb|math.nan| en la respuesta dada por el sistema, entonces el sistema pone buena la pregunta.

\item \verb|<tipo = respuesta corta[, funcion=_txtfun_]>|
Es posible pasar una funci\'on que pre-procese las respuestas. Lo \'unico es que hay que pasarla como el texto de una funci\'on. Digamos que queremos calificar una pregunta cuya respuesta es una ecuaci\'on lineal $mx+b$.
\begin{verbatim}
<variables>
m, b = randint(1, 9)
ftxt = r"lambda ss: set(ss.replace('+', ' ').split())"
resp = {txt.coef(m) + 'x', str(b)}
...
<item, f=ftxt>
resp
\end{verbatim}
Este proceso califica buenas preguntas que se hayan escrito de la forma $mx+b$ o $b+mx$, al hacer la comparaci\'on utilizando conjuntos.
\end{itemize}

\subsubsection{Consideraciones sobre el error}
En general, es complicado determinar el rango de error admitido en una pregunta de respuesta corta cuando la respuesta \emph{no} es un número entero. Por ejemplo Moodle utiliza 2 técnicas distintas. Una opción que es 'Error nominal', el cual es una medida de error absoluto. El inconveniente que tiene en una pregunta parametrizada, es que si todas las respuestas no están en el mismo rango, es decir, entre $1.00\times 10^n$ y $9.99\times 10^n$, con $n\in\mathbb Z$ fijo, la exigencia de la respuesta sería distinto, dependiendo del valor de $n$. La otra forma que tiene es el error relativo, que al menos, sin profundizar mucho si es o no la forma correcta de evaluar, tiene un comportamiento que se considera no deseado, y es el siguiente:

Si la respuesta correcta es 1.38532 y la persona estudiante responde 1.38 (que está mal redondeado), se tiene un error del 0.43\%. Si se utiliza un 0.2\% de error válido, esta respuesta sería calificada como mala. Por otro lado, si la respuesta correcta es 9.38532 y la persona estudiante responde con 9.38, la respuesta dada tendría un error del 0.06\%, y la respuesta se tomaría como buena.

Esta plataforma implementa un error nominal pero en término de cifras significativas. Aceptar por ejemplo un 0.2\% de error, es definir el error en la pregunta igual a 0.01. Lo que hace es tomar la mantisa de la respuesta, y sumar y restar el error dado. Si la respuesta dada está en dicho rango, se toma como buena.

¿Qué se debe considerar al respecto? Que al momento de presentar el reporte a la persona estudiante, se da la respuesta redondeada que se esperaba. Volviendo a nuestro ejemplo anterior, con 1.38532 y un error de 0.01, se toma el rango de la repuesta correcta entre [1.37532, 1.39532]. Al usuario se le muestra como 1.39 la respuesta correcta. Si responde 1.38 tiene la respuesta correcta. Si responde 1.40 tiene la respuesta mala.

\section{Informaci\'on general}

\subsection{¿Cómo construir una sucesión?}

En python las sucesiones para las funciones \verb|choice|, \verb|choices| o \verb|sample| se pueden definir mediante una de las siguientes formas:
\begin{itemize}
  \item \verb|range(<stop>)|
  \item \verb|range(<start>, <stop>[, <step>])|
\end{itemize}

%En el caso de un solo argumento, entonces la sucesión comienza en 0 y termina en \verb|stop - 1|. Si tiene dos argumentos, entonces comienza en \verb|start| y finaliza en \verb|stop-1|. Con tres argumentos la función \verb|range| define la sucesión \verb|start|, \verb|start + step|, \verb|start + 2*step|, \dots, \verb|start + k*step|, donde \verb|start + k*step < stop <= start + (k+1)*step|.

\begin{enumerate}
  \item Si la sucesión está dada por los $j$ elementos $0, 1, 2, \dots, j - 1$, se construye con \verb|xs = range(j)|.
  \item Si la sucesión está dada por los $j$ elementos: $i, i+1, i+2, \dots, i + j - 1$, se construye con \verb|xs = range(i, i+j)|.
  \item Para una sucesión aritmética de $k$ elementos $i, i + d, i + 2d, \dots, i + (k-1)d$, se construye con \verb|xs = range(i, j, d)|, donde $i + (k-1)d < j \leq i + kd$.
  \item Para concatenar dos o más sucesiones:\\[1ex]
    \verb|xs = [*range(<start>,<stop>[,<step>]), *range(<start>,<stop>[,<step>])]|.
\end{enumerate}

Si lo que se quiere es tomar un elemento aleatorio de una sucesión simple, entonces mejor utilizar las funciones \verb|randrange| o \verb|randint|. 

\subsection{Constantes y funciones generales}
Lo que se tiene es un subconjunto de constantes y funciones de python, y algunas programadas específicamente para la generación de pruebas.

%% \begin{funcion}
%% \verb|div|; \verb|inf|; \verb|pi| \\[1ex]
%%   El valor de $\pi$, y constantes utilizadas para evaluar límites en respuesta corta. Así $-\infty$ se escribiría \verb|-inf| y $+\infty$ como \verb|inf| o \verb|+inf|. Si el límite es divergente entonces utilizar \verb|div| (internamente se trabaja con la constante \verb|nan|, \emph{not a number}).
%% 
%%   La constante \verb|e| se incluye como parte de la biblioteca de matemática: \verb|math.e|.
%% \end{funcion}

\begin{funcion}
  \verb|lambda x: __funcion que depende de x__| \\[1ex]
  Funciones anónimas. \textbf{No debe tener variables libres}, es decir, todas las variables deben definirse como argumento. Por ejemplo, para definir una función cuadrática con parámetros $a$, $b$, $c$ generados de manera aleatoria, se podría hacer así:
\small
\begin{verbatim}
<tipo = respuesta corta>

<variables>
a, b, c = sample([*range(-9, 0), *range(1, 10)], 3)
cuad = lambda x, a, b, c: a * x**2 + b * x + c

x = randint(-3, 3)

<pregunta>
Considere $f(x) = @[txt.coef(a)]x^2 @[txt.coef(b, True)]x @[txt.conSigno(c)]$. 
Determine $f(@[x])$.

<item>
cuad(x, a, b, c)
\end{verbatim}
\normalsize

Esta implementación no permite el uso de funciones recursivas. Sin embargo se puede utilizar un combinador-Y para lograrlo. Ello consiste en agregar una función como parámetro, y luego, a la hora de llamar a la función, pasarse a sí misma como argumento. Veamos un ejemplo:

\small
\begin{verbatim}
<tipo = respuesta corta>

<variables>
i = randrange(3)
h = [math.cos, math.sin, lambda x: math.exp(-x)][i]
txt = ['\\cos(x)', '\\sin(x)', 'e^{-x}'][i]
x0 = [0, 1, 2][i]
n = randint(2, 5)

pf = lambda g, f, x0, n: x0 if n == 0 else g(g, f, f(x0), n - 1)

<pregunta>
Determine el valor de realizar @[n] iteraciones del punto fijo de la función 
$f(x)=@[txt]$ iniciando en @[x0].

<item, error = 0.01>
pf(pf, h, x0, n)
\end{verbatim}
\normalsize

\end{funcion}

\begin{funcion}
  \verb|len(xs)|, \verb|sorted(xs)| \\[1ex]
  Devuelven respectivamente la longitud de un iterable, y el iterable de manera ordenada creciente. Si se quiere ordenar \verb|xs| de manera decreciente, entonces se debe escribir \verb|sorted(xs, reverse = True)|. También se puede utilizar una función para especificar la función de orden: \newline
  \verb|sorted(xs, key = __funcion__)|. Por ejemplo:
  \small
  \begin{verbatim}
  ls = [*range(-9, 10)]
  ls = sorted(sample(ls, 5), key = abs)
  \end{verbatim}
  \normalsize
  podría dar como resultado \verb|[2, 4, -5, -6, 8]| para \verb|ls|.
\end{funcion}

\begin{funcion} \verb|__exp_True__ if __exp_bool__ else __exp_False__| \\[1ex]
  Un \verb|if| en un sólo renglón. En las expresiones booleanas se puede utilizar \verb|not|, \verb|or| y \verb|and| según se necesite.
\end{funcion}

%% \begin{funcion}
%%   {\small
%% \verb|txt.fraccion(num, den=1, conSigno=False, signoNum=False, dfrac=True, arg='', coef=False)|} \\[1ex]
%% Simplifica e imprime utilizando \verb|dfrac|. De manera opcional se especifica si se obliga el signo \verb|+|; si el signo se imprime en el numerador o se imprime afuera; si se utiliza \verb|tfrac| en lugar de \verb|dfrac|, y si tiene un argumento, de manera que si da 1 o $-1$ no se imprima el valor. Se puede utilizar \verb|Fraction| en el numerador, y ya no ser\'ia necesario especificar el denominador. Si \verb|coef=True|, entonces se imprime sólo el signo si es 1 o $-1$.
%% 
%% Si el argumento es distinto de vac\'io, el numerador se toma como un coeficiente.
%% \end{funcion}
%% 
%% \begin{funcion}
%%   \verb|txt.raiz(a[, n=2[, conSigno=False]])| \\[1ex]
%%   Simplifica y escribe la raíz respectiva en \LaTeX. \verb|conSigno| es un booleano (\verb|False| o \verb|True| con \verb|False| como predeterminado) que escribe un signo + antes de la raíz si es positiva.
%% \end{funcion}
%% 
%% \begin{funcion}
%%   \verb|txt.coef(a[, conSigno=False])| \\[1ex]
%%   Coeficiente que precede a una variable entera. Si $a=1$ entonces no se escribe (o se escribe solo un signo + si \verb|conSigno|). De manera similar ocurre si $a=-1$. En caso contrario imprime el valor.
%% \end{funcion}
%% 
%% \begin{funcion}
%%   \verb|txt.conSigno(n)| \\[1ex]
%%   Imprime el número entero con signo.
%% \end{funcion}
%% 
%% \begin{funcion}
%%   \verb|txt.expo(n)| \\[1ex]
%%   Si \verb|n == 1| no imprime nada. Si no imprime \verb|^{n}|.
%% \end{funcion}
%% 
%% \begin{funcion}
%%   \verb|txt.numero(n)| \\[1ex]
%%   Para $n > 0$, devuelve el texto que representa al valor de $n$.
%% \end{funcion}
%% 
%% \begin{funcion}
%%   \verb|txt.decimal(n, cifras, conSigno = False)| \\[1ex]
%%   Imprimir un número según las cifras dadas.
%% \end{funcion}

\subsection{Trabajando con sucesiones}

En python las sucesiones son 0 indexadas. Así, si \verb|xs| es una sucesión, entonces \verb|xs[0]| es el primer elemento. El tamaño de una sucesión se da por \verb|len(xs)|. El último elemento sería \verb|xs[-1]|. Un subconjunto: \verb|xs[i:j]| va a ser los elementos que hay desde \verb|xs[i]| hasta \verb|xs[j-1]|.

\subsection{Listas por comprensión}
\small
\begin{verbatim}
xs = [i for i in range(10)]
ys = [random() for i in range(10)]
ls = sorted([(xs[i], ys[i]) for i in range(10)], key = lambda tt: tt[1])
\end{verbatim}
\normalsize

\noindent Se pueden agregar condicionales:
\small
\begin{verbatim}
siglos_bisiestos = [i for i in range(1000, 3001, 100) if i % 100 > 0 or i % 400 == 0]
\end{verbatim}
\normalsize

\subsection{Conjuntos}
Se construyen utilizando valores entre llaves, o con el comando \verb|set| sobre una lista.
\small
\begin{verbatim}
s1 = set(sample(range(1, 16), 5))
s2 = {3, 5, 7, 11}
\end{verbatim}
\normalsize

\subsubsection{Operaciones}
\begin{funcion}
  \verb|ss.difference(s1, s2, ...)|; \verb|ss.intersection(s1, s2, ...)|; \verb|ss.union(s1, s2, ...)| \\[1ex]
  Se resta a \verb|ss| cada uno de los conjuntos $s_i$ y se devuelve el resultado; devuelve la intersección de los conjuntos \verb|ss|, \verb|s1|, \dots; y lo mismo para la unión.
\end{funcion}

\begin{funcion}
  \verb|ss.symmetric_difference(s1)| \\[1ex]
  Diferencia simétrica de \verb|ss| con \verb|s1|.
\end{funcion}

\begin{funcion}
  \verb|ss.isdisjoint(s1)|; \verb|ss.issubset(s1)| \\[1ex]
  \verb|True| si \verb|ss| y \verb|s1| son disjuntos; \verb|True| si \verb|ss| es subconjunto de \verb|s1|.
\end{funcion}

\subsection{Formato de texto}
\label{formato}

\begin{itemize}
  \item Si \verb|num| es un entero, se puede imprimir el entero con signo como: \verb|'%+d' % num|
  \item Si \verb|x| es un número con decimales:
    \begin{itemize}
      \item \verb|'%f' % x| imprime el número con 6 decimales de manera predeterminada.
      \item \verb|'%.nf' % x|, donde \verb|n| es un entero positivo, imprime el número con \verb|n| decimales.
      \item \verb|'%e' % x| imprime el número en notación científica con 6 decimales de forma predeterminada.
      \item \verb|'%.ne' % x|, donde \verb|n| es un entero positivo, imprime el número en notación científica con \verb|n| decimales.
    \end{itemize}
  \item Si se necesita concatenar varios textos: \verb|'(%d, %d)' % (m, n)| representa un par ordenado de enteros.
\end{itemize}

\section{Ejemplos}
\begin{ejem}
Obtener 3 valores distintos $a,b,c$, entre -9 y 9, quitando el cero:
\begin{verbatim}
a, b, c = sample([*range(-9, 0), *range(1, 10)], 3)
\end{verbatim}

Y si se quisiera que $|a|\neq |b| \neq |c|$:
\begin{verbatim}
a, b, c = sample(range(1, 10), 3)
a = choice([-a, a])
b = choice([-b, b])
c = choice([-c, c])
\end{verbatim}

Si se quiere que sea al menos uno negativo y a lo sumo dos negativos:
\begin{verbatim}
a, b, c = sample(range(1, 10), 3)
sa, sb, sc = sample([-1, -1, 1, 1], 3)
a = a * sa
b = b * sb
c = c * sc
\end{verbatim}
y si se ocuparan 2 negativos y uno positivo, cambiar la generaci\'on de signos por: \\
\verb|sa, sb, sc = sample([-1, -1, 1], 3)|
\end{ejem}

\begin{ejem}
  Se quiere parametrizar una funci\'on $f$, mostrar el texto, y evaluarla en un valor entre 0.5 y 1, tambi\'en parametrizado:
\begin{verbatim}
opc = randrange(3)
txtf = [r'\exp(x)', r'\ln(x)', r'1-\sen^2(x)'][opc]
ff   = [math.exp, math.log, lambda x: 1 - (math.sin(x))**2][opc]
x = 0.01 * randint(50, 99)
fx = f(x)
\end{verbatim}
luego en el texto, como sabemos que $x$ tiene como m\'aximo 2 decimales despu\'es del 0, podemos usar:
\verb|@['%.2f' % x]|
\end{ejem}

\begin{ejem}
  Para imprimir $ax^2+bx+c$ donde $b$ y $c$ podr\'ian ser 0 (entonces no queremos que aparezcan): \\
  \small
  \verb|@[txt.coef(a)]x^2 @[txt.coef(b, conSigno=True, arg='x')] @[txt.coef(c, conSigno=True, arg=' ']| \\
  \normalsize
  Observe que en el caso de $c$, se le pasa como argumento un espacio en blanco. Con eso aseguramos que en caso de ser 0 no lo pone, y a\'un as\'i, si es 1, s\'i escribe el valor.
\end{ejem}

\begin{ejem}
  Se quiere escribir $a + \dfrac{p}{q + \sqrt{r}}$, donde $p$ puede ser positivo o negativo, queremos que $r$ se pueda simplificar, tanto sacar t\'erminos fuera de la ra\'iz como simplificar, junto con $q$, con el valor de $p$.

  En las variables, definir:
\small
\begin{verbatim}
signo = -1 if p < 0 else 1
p = abs(p)
coef, argu = mate.raiz(r)
factor = math.gcd(q, coef) if argu > 1 else q + coef
q = q // factor if argu > 1 else factor
coef = coef // factor if argu > 1 else 0
ff = Fraction(p, factor)
exp = r'%d %s \dfrac{%d}{%d %s\sqrt{%d}}' % (a, txt.coef(signo, conSigno=True), ...
      ff.numerator, ff.denominator * q, txt.coef(ff.denominator * coef, conSigno=True, arg) ...
      if argu > 1 else r'%d %s' % (a, txt.fraccion(ff, conSigno=True))
\end{verbatim}
\normalsize
\end{ejem}
Los puntos suspensivos \textbf{no} se usan. Se debe escribir toda la expresi\'on en un solo rengl\'on. Luego es suficiente escribir en el documento \verb|$@[exp]$|.
\end{document}

%%%%%%%%%%%%%%%%%%%%%%%%%%%%%%%%%%%%%%%%%%%%%%%%%%%%%%%%%%%%%%%%%%%%%%%%%%%%%%%%%%%%%%%%%%%%%%%


\section{Funciones}

<variables>
  __nombre__ = __definicion__

    Varias opciones, indicadas por \verb|<item>|. Al finalizar las opciones, entonces en la etiqueta  \verb|<resp>| aparece el índice de la respuesta correcta. Si no viene la respuesta correcta, entonces se asume que el primer item es el correcto.

    Para saber cómo ordenar, entonces se puede agregar a la etiqueta de \\[1ex]
    \verb|<tipo = seleccion unica, opciones = <opc>>|, donde para \verb|<opc>| se tienen:
    \begin{description}
      \item [\texttt{fijas}:]   Mantiene el orden dado por el usuario.
      \item [\texttt{indices}:] Cada \verb|<item>| trae el índice que le corresponde.
      \item [\texttt{ordenar}:] Ordena las opciones de menor a mayor.
      \item [\texttt{random}:]  Ordena las opciones de forma aleatoria.
    \end{description}
\end{itemize}

